% Options for packages loaded elsewhere
\PassOptionsToPackage{unicode}{hyperref}
\PassOptionsToPackage{hyphens}{url}
%
\documentclass[
]{article}
\usepackage{lmodern}
\usepackage{amssymb,amsmath}
\usepackage{ifxetex,ifluatex}
\ifnum 0\ifxetex 1\fi\ifluatex 1\fi=0 % if pdftex
  \usepackage[T1]{fontenc}
  \usepackage[utf8]{inputenc}
  \usepackage{textcomp} % provide euro and other symbols
\else % if luatex or xetex
  \usepackage{unicode-math}
  \defaultfontfeatures{Scale=MatchLowercase}
  \defaultfontfeatures[\rmfamily]{Ligatures=TeX,Scale=1}
\fi
% Use upquote if available, for straight quotes in verbatim environments
\IfFileExists{upquote.sty}{\usepackage{upquote}}{}
\IfFileExists{microtype.sty}{% use microtype if available
  \usepackage[]{microtype}
  \UseMicrotypeSet[protrusion]{basicmath} % disable protrusion for tt fonts
}{}
\makeatletter
\@ifundefined{KOMAClassName}{% if non-KOMA class
  \IfFileExists{parskip.sty}{%
    \usepackage{parskip}
  }{% else
    \setlength{\parindent}{0pt}
    \setlength{\parskip}{6pt plus 2pt minus 1pt}}
}{% if KOMA class
  \KOMAoptions{parskip=half}}
\makeatother
\usepackage{xcolor}
\IfFileExists{xurl.sty}{\usepackage{xurl}}{} % add URL line breaks if available
\IfFileExists{bookmark.sty}{\usepackage{bookmark}}{\usepackage{hyperref}}
\hypersetup{
  pdftitle={Homework 1},
  pdfauthor={BIOF 339},
  hidelinks,
  pdfcreator={LaTeX via pandoc}}
\urlstyle{same} % disable monospaced font for URLs
\usepackage[margin=1in]{geometry}
\usepackage{color}
\usepackage{fancyvrb}
\newcommand{\VerbBar}{|}
\newcommand{\VERB}{\Verb[commandchars=\\\{\}]}
\DefineVerbatimEnvironment{Highlighting}{Verbatim}{commandchars=\\\{\}}
% Add ',fontsize=\small' for more characters per line
\usepackage{framed}
\definecolor{shadecolor}{RGB}{248,248,248}
\newenvironment{Shaded}{\begin{snugshade}}{\end{snugshade}}
\newcommand{\AlertTok}[1]{\textcolor[rgb]{0.94,0.16,0.16}{#1}}
\newcommand{\AnnotationTok}[1]{\textcolor[rgb]{0.56,0.35,0.01}{\textbf{\textit{#1}}}}
\newcommand{\AttributeTok}[1]{\textcolor[rgb]{0.77,0.63,0.00}{#1}}
\newcommand{\BaseNTok}[1]{\textcolor[rgb]{0.00,0.00,0.81}{#1}}
\newcommand{\BuiltInTok}[1]{#1}
\newcommand{\CharTok}[1]{\textcolor[rgb]{0.31,0.60,0.02}{#1}}
\newcommand{\CommentTok}[1]{\textcolor[rgb]{0.56,0.35,0.01}{\textit{#1}}}
\newcommand{\CommentVarTok}[1]{\textcolor[rgb]{0.56,0.35,0.01}{\textbf{\textit{#1}}}}
\newcommand{\ConstantTok}[1]{\textcolor[rgb]{0.00,0.00,0.00}{#1}}
\newcommand{\ControlFlowTok}[1]{\textcolor[rgb]{0.13,0.29,0.53}{\textbf{#1}}}
\newcommand{\DataTypeTok}[1]{\textcolor[rgb]{0.13,0.29,0.53}{#1}}
\newcommand{\DecValTok}[1]{\textcolor[rgb]{0.00,0.00,0.81}{#1}}
\newcommand{\DocumentationTok}[1]{\textcolor[rgb]{0.56,0.35,0.01}{\textbf{\textit{#1}}}}
\newcommand{\ErrorTok}[1]{\textcolor[rgb]{0.64,0.00,0.00}{\textbf{#1}}}
\newcommand{\ExtensionTok}[1]{#1}
\newcommand{\FloatTok}[1]{\textcolor[rgb]{0.00,0.00,0.81}{#1}}
\newcommand{\FunctionTok}[1]{\textcolor[rgb]{0.00,0.00,0.00}{#1}}
\newcommand{\ImportTok}[1]{#1}
\newcommand{\InformationTok}[1]{\textcolor[rgb]{0.56,0.35,0.01}{\textbf{\textit{#1}}}}
\newcommand{\KeywordTok}[1]{\textcolor[rgb]{0.13,0.29,0.53}{\textbf{#1}}}
\newcommand{\NormalTok}[1]{#1}
\newcommand{\OperatorTok}[1]{\textcolor[rgb]{0.81,0.36,0.00}{\textbf{#1}}}
\newcommand{\OtherTok}[1]{\textcolor[rgb]{0.56,0.35,0.01}{#1}}
\newcommand{\PreprocessorTok}[1]{\textcolor[rgb]{0.56,0.35,0.01}{\textit{#1}}}
\newcommand{\RegionMarkerTok}[1]{#1}
\newcommand{\SpecialCharTok}[1]{\textcolor[rgb]{0.00,0.00,0.00}{#1}}
\newcommand{\SpecialStringTok}[1]{\textcolor[rgb]{0.31,0.60,0.02}{#1}}
\newcommand{\StringTok}[1]{\textcolor[rgb]{0.31,0.60,0.02}{#1}}
\newcommand{\VariableTok}[1]{\textcolor[rgb]{0.00,0.00,0.00}{#1}}
\newcommand{\VerbatimStringTok}[1]{\textcolor[rgb]{0.31,0.60,0.02}{#1}}
\newcommand{\WarningTok}[1]{\textcolor[rgb]{0.56,0.35,0.01}{\textbf{\textit{#1}}}}
\usepackage{graphicx,grffile}
\makeatletter
\def\maxwidth{\ifdim\Gin@nat@width>\linewidth\linewidth\else\Gin@nat@width\fi}
\def\maxheight{\ifdim\Gin@nat@height>\textheight\textheight\else\Gin@nat@height\fi}
\makeatother
% Scale images if necessary, so that they will not overflow the page
% margins by default, and it is still possible to overwrite the defaults
% using explicit options in \includegraphics[width, height, ...]{}
\setkeys{Gin}{width=\maxwidth,height=\maxheight,keepaspectratio}
% Set default figure placement to htbp
\makeatletter
\def\fps@figure{htbp}
\makeatother
\setlength{\emergencystretch}{3em} % prevent overfull lines
\providecommand{\tightlist}{%
  \setlength{\itemsep}{0pt}\setlength{\parskip}{0pt}}
\setcounter{secnumdepth}{-\maxdimen} % remove section numbering

\title{Homework 1}
\author{BIOF 339}
\date{}

\begin{document}
\maketitle

\hypertarget{installation-and-setup}{%
\section{Installation and setup}\label{installation-and-setup}}

\begin{enumerate}
\def\labelenumi{\arabic{enumi}.}
\tightlist
\item
  Install R and RStudio on your computer following the instructions in
  class. If you have trouble, reach out to me on Slack
\item
  Install the \textbf{pacman} package by running the following R command
  in the RStudio console pane: \texttt{install.packages("pacman")}. Make
  sure of spelling and case.
\item
  Now install a few more packages using the \textbf{pacman} package.
  This simplifies some aspects of package installation and loading. Open
  a new script window and type the following code in that window:
\end{enumerate}

\begin{Shaded}
\begin{Highlighting}[]
\KeywordTok{library}\NormalTok{(pacman)}
\KeywordTok{p_load}\NormalTok{(}\DataTypeTok{char =} \KeywordTok{c}\NormalTok{(}\StringTok{"tidyverse"}\NormalTok{, }\StringTok{'broom'}\NormalTok{, }\StringTok{'janitor'}\NormalTok{, }\StringTok{'readxl'}\NormalTok{))}
\end{Highlighting}
\end{Shaded}

This installs and loads the meta-package \textbf{tidyverse}, and the
packages \textbf{broom}, \textbf{janitor} and \textbf{readxl} into R.

\begin{infobox}

\texttt{p\_load} is a function within the \textbf{pacman} package. You
can think of functions as recipes and packages as recipe books, if that
helps.\\
The \texttt{p\_load} function installs a package if you don't have it on
your computer, and then loads it. It just loads the package if you
already have it installed. Note that you only need to install a package
\textbf{once} on a computer.

\end{infobox}

\begin{infobox}

Note that I've interchanged single and double quotes in the code
snippet. Please feel free to use either single- or double-quotes or a
mixture, as long as the quotes are properly paired by type.

\end{infobox}

\hypertarget{r-markdown-practice-10-pts}{%
\section{R Markdown practice (10
pts)}\label{r-markdown-practice-10-pts}}

We will continue with the \texttt{airquality} dataset we worked with in
class. Learn more about this dataset by typing either
\texttt{help(airquality)} or \texttt{?airquality} at the console prompt.

In this homework you will create a report and a presentation on this
dataset using RMarkdown. You will do this in the same RStudio project
you created this week.

Your documents will incorporate the following 3 code snippets (in order)
as R chunks and one piece of code inline. If I'm calling a package you
do not have installed, use the \texttt{p\_load} function above to
install it.

\hypertarget{snippet-1}{%
\subsubsection{Snippet 1}\label{snippet-1}}

\begin{Shaded}
\begin{Highlighting}[]
\KeywordTok{library}\NormalTok{(tidyverse)}
\KeywordTok{library}\NormalTok{(knitr)}
\NormalTok{avg_temp_by_month <-}\StringTok{ }\NormalTok{airquality }\OperatorTok\StringTok{ }
\StringTok{  }\KeywordTok{group_by}\NormalTok{(Month) }\OperatorTok\StringTok{ }
\StringTok{  }\KeywordTok{summarize}\NormalTok{(}\DataTypeTok{avgTemp =} \KeywordTok{mean}\NormalTok{(Temp, }\DataTypeTok{na.rm=}\NormalTok{T))}
\KeywordTok{kable}\NormalTok{(avg_temp_by_month)}
\end{Highlighting}
\end{Shaded}

\hypertarget{snippet-2}{%
\subsubsection{Snippet 2}\label{snippet-2}}

\begin{Shaded}
\begin{Highlighting}[]
\KeywordTok{ggplot}\NormalTok{(avg_temp_by_month, }\KeywordTok{aes}\NormalTok{(}\DataTypeTok{x =}\NormalTok{ Month, }\DataTypeTok{y =}\NormalTok{ avgTemp)) }\OperatorTok{+}
\StringTok{  }\KeywordTok{geom_point}\NormalTok{() }\OperatorTok{+}\StringTok{ }
\StringTok{  }\KeywordTok{geom_line}\NormalTok{(}\DataTypeTok{color =} \StringTok{'blue'}\NormalTok{) }\OperatorTok{+}\StringTok{ }
\StringTok{  }\KeywordTok{labs}\NormalTok{(}\DataTypeTok{y =} \StringTok{'Average Temperature (F)'}\NormalTok{)}
\end{Highlighting}
\end{Shaded}

\hypertarget{snippet-3}{%
\subsubsection{Snippet 3}\label{snippet-3}}

\begin{Shaded}
\begin{Highlighting}[]
\KeywordTok{ggplot}\NormalTok{(airquality, }\KeywordTok{aes}\NormalTok{(}\DataTypeTok{x =}\NormalTok{ Wind, }\DataTypeTok{y =}\NormalTok{ Ozone)) }\OperatorTok{+}\StringTok{ }
\StringTok{  }\KeywordTok{geom_point}\NormalTok{() }\OperatorTok{+}\StringTok{ }
\StringTok{  }\KeywordTok{geom_smooth}\NormalTok{(}\DataTypeTok{color =} \StringTok{'blue'}\NormalTok{, }\DataTypeTok{se =} \OtherTok{FALSE}\NormalTok{)}
\end{Highlighting}
\end{Shaded}

\hypertarget{inline-snippet}{%
\subsubsection{Inline snippet}\label{inline-snippet}}

\texttt{max(airquality\$Ozone,\ na.rm\ =\ T)}, which is the maximum
recorded Ozone level. Incorporate into a sentence.

\hypertarget{markdown}{%
\subsubsection{Markdown}\label{markdown}}

There is a Markdown Quick Reference available in under the Help menu to
get you started.

\begin{center}\rule{0.5\linewidth}{0.5pt}\end{center}

\hypertarget{assignment-1}{%
\section{Assignment 1}\label{assignment-1}}

\hypertarget{background-reading}{%
\subsection{Background reading}\label{background-reading}}

\begin{itemize}
\tightlist
\item
  Read the first 3 sections of the document at
  \url{https://rmarkdown.rstudio.com/lesson-1.html}.
\item
  To go into more detail on each style below see:

  \begin{itemize}
  \tightlist
  \item
    HTML document:
    \url{https://bookdown.org/yihui/rmarkdown/html-document.html}
  \item
    HTML ioslide presentation:
    \url{https://bookdown.org/yihui/rmarkdown/ioslides-presentation.html}
  \end{itemize}
\end{itemize}

\hypertarget{work-10-pts}{%
\subsection{Work (10 pts)}\label{work-10-pts}}

\begin{enumerate}
\def\labelenumi{\arabic{enumi}.}
\tightlist
\item
  Write a RMarkdown report incorporating these code snippets and create
  a story around them. You will set the output to HTML. Call this file
  \texttt{\textless{}your\ name\textgreater{}\_HW1\_report.Rmd}
\item
  Write a RMarkdown presentation incorporating these same code snippets.
  The output format should be HTML(ioslides). Call this file
  \texttt{\textless{}your\ name\textgreater{}\_HW1\_slides.Rmd}
\end{enumerate}

Submit both the Rmds and the corresponding HTML files to Canvas.

\hypertarget{assignment-2-10-points}{%
\section{Assignment 2 (10 points)}\label{assignment-2-10-points}}

The following section contains a templated R Markdown file that you can
copy into a fresh R Markdown document in RStudio. You are expected to
use the help system in R/RStudio as well as Google, if need be, to fill
in the blanks below.

\begin{infobox}

\begin{enumerate}
\def\labelenumi{\arabic{enumi}.}
\tightlist
\item
  The symbol \texttt{==} checks for equality between two objects, and
  returns \texttt{TRUE} or \texttt{FALSE}
\item
  Just as we saw \texttt{is.character} and \texttt{is.numeric}, you can
  check missing values with \texttt{is.na}, which gives a \texttt{TRUE}
  everytime it encounters a missing value in a data array, and
  \texttt{FALSE} otherwise. Internally in R, \texttt{TRUE} = 1.
\end{enumerate}

\end{infobox}

\begin{center}\rule{0.5\linewidth}{0.5pt}\end{center}

\begin{Shaded}
\begin{Highlighting}[]
\NormalTok{---}
\NormalTok{title: Homework 1, Part 2}
\NormalTok{author: _________ ___________}
\NormalTok{date: "BIOF 339"}
\NormalTok{---}


\BaseNTok{```\{r, echo = FALSE, eval=TRUE\}}
\BaseNTok{knitr::opts_chunk$set(message=FALSE, warning=FALSE)}
\BaseNTok{```}


\FunctionTok{## Descriptive statistics}

\NormalTok{We'll start with the }\BaseNTok{`airquality`}\NormalTok{ data set that is in-built in R. }

\NormalTok{1. }\StringTok{The average temperature in June was }\BaseNTok{`r ______(airquality$_____[airquality$Month==6], na.rm=TRUE)`}\StringTok{.}
\StringTok{2. Solar radiation data is missing on }\BaseNTok{`r sum(is._____(airquality[,"Solar.R"]))`}\StringTok{ days, or in }\BaseNTok{`r 100 * sum(is._____(airquality[,"Solar.R"]))/_______(airquality)`}\StringTok{ percent of all the days collected. }

\NormalTok{We can also visualize the missing data patterns in this data set.}

\BaseNTok{```\{r, echo = TRUE, eval=TRUE\}}
\BaseNTok{library(pacman)}
\BaseNTok{p_load('naniar') # This is a package for missing data}
\BaseNTok{vis_miss(______________) # see the documentation for vis_miss}
\BaseNTok{```}

\NormalTok{Let's grab a more interesting data set. We will download and use the }\OtherTok{[Palmer Station penguins data set](doi:10.1371/journal.pone.0090081)}\NormalTok{, which is in the form }
\NormalTok{of an R package on GitHub. }

\BaseNTok{```\{r, echo = TRUE, eval=TRUE\}}
\BaseNTok{library(pacman)}
\BaseNTok{p_______('visdat') # Install and load visdat}
\BaseNTok{p_install_gh('allisonhorst/palmerpenguins') }
\BaseNTok{p_load('palmerpenguins')}
\BaseNTok{vis_dat(penguins)}
\BaseNTok{```}
\end{Highlighting}
\end{Shaded}

\begin{center}\rule{0.5\linewidth}{0.5pt}\end{center}

Icons made by Freepik from www.flaticon.com

\end{document}

% Options for packages loaded elsewhere
\PassOptionsToPackage{unicode}{hyperref}
\PassOptionsToPackage{hyphens}{url}
%
\documentclass[
]{article}
\usepackage{lmodern}
\usepackage{amssymb,amsmath}
\usepackage{ifxetex,ifluatex}
\ifnum 0\ifxetex 1\fi\ifluatex 1\fi=0 % if pdftex
  \usepackage[T1]{fontenc}
  \usepackage[utf8]{inputenc}
  \usepackage{textcomp} % provide euro and other symbols
\else % if luatex or xetex
  \usepackage{unicode-math}
  \defaultfontfeatures{Scale=MatchLowercase}
  \defaultfontfeatures[\rmfamily]{Ligatures=TeX,Scale=1}
\fi
% Use upquote if available, for straight quotes in verbatim environments
\IfFileExists{upquote.sty}{\usepackage{upquote}}{}
\IfFileExists{microtype.sty}{% use microtype if available
  \usepackage[]{microtype}
  \UseMicrotypeSet[protrusion]{basicmath} % disable protrusion for tt fonts
}{}
\makeatletter
\@ifundefined{KOMAClassName}{% if non-KOMA class
  \IfFileExists{parskip.sty}{%
    \usepackage{parskip}
  }{% else
    \setlength{\parindent}{0pt}
    \setlength{\parskip}{6pt plus 2pt minus 1pt}}
}{% if KOMA class
  \KOMAoptions{parskip=half}}
\makeatother
\usepackage{xcolor}
\IfFileExists{xurl.sty}{\usepackage{xurl}}{} % add URL line breaks if available
\IfFileExists{bookmark.sty}{\usepackage{bookmark}}{\usepackage{hyperref}}
\hypersetup{
  pdftitle={BIOF 339: Practical R},
  hidelinks,
  pdfcreator={LaTeX via pandoc}}
\urlstyle{same} % disable monospaced font for URLs
\usepackage[margin=1in]{geometry}
\usepackage{longtable,booktabs}
% Correct order of tables after \paragraph or \subparagraph
\usepackage{etoolbox}
\makeatletter
\patchcmd\longtable{\par}{\if@noskipsec\mbox{}\fi\par}{}{}
\makeatother
% Allow footnotes in longtable head/foot
\IfFileExists{footnotehyper.sty}{\usepackage{footnotehyper}}{\usepackage{footnote}}
\makesavenoteenv{longtable}
\usepackage{graphicx,grffile}
\makeatletter
\def\maxwidth{\ifdim\Gin@nat@width>\linewidth\linewidth\else\Gin@nat@width\fi}
\def\maxheight{\ifdim\Gin@nat@height>\textheight\textheight\else\Gin@nat@height\fi}
\makeatother
% Scale images if necessary, so that they will not overflow the page
% margins by default, and it is still possible to overwrite the defaults
% using explicit options in \includegraphics[width, height, ...]{}
\setkeys{Gin}{width=\maxwidth,height=\maxheight,keepaspectratio}
% Set default figure placement to htbp
\makeatletter
\def\fps@figure{htbp}
\makeatother
\setlength{\emergencystretch}{3em} % prevent overfull lines
\providecommand{\tightlist}{%
  \setlength{\itemsep}{0pt}\setlength{\parskip}{0pt}}
\setcounter{secnumdepth}{-\maxdimen} % remove section numbering

\title{BIOF 339: Practical R}
\author{}
\date{\vspace{-2.5em}}

\begin{document}
\maketitle

\hypertarget{instructor}{%
\subsection{Instructor}\label{instructor}}

Abhijit Dasgupta\\
Contact: via Slack or email
(\href{mailto:adasgupta+biof339@araastat.com}{\nolinkurl{adasgupta+biof339@araastat.com}})

\hypertarget{course-description}{%
\subsection{Course description}\label{course-description}}

The goal of this course is to introduce R as an analysis platform and
tool for data science rather than a programming language. Throughout the
course, emphasis will be placed on example-driven learning. Topics to be
covered include: installation of R and R packages; command line R; R
data types; loading data in R; manipulating data; exploring data through
visualization; statistical tests; correcting for multiple comparisons;
building models; generating publication-quality graphics; creating
reports using RMarkdown. No prior programming experience is required.

\hypertarget{learning-objectives}{%
\section{Learning Objectives}\label{learning-objectives}}

\begin{itemize}
\tightlist
\item
  Run R and RStudio, making use of inherent R features
\item
  Find and make use of the extensive packages (R add-ons) available for
  analyzing biological and other forms of data
\item
  Load, manipulate, and combine data to make it amenable to further
  analyses
\item
  Visualize data with extensive graphics capabilities of R (including
  ggplot)
\item
  Use R to run statistical models and hypothesis tests and report
  results conforming to standards expected in scientific journals
\item
  Write reports using the powerful \texttt{rmarkdown} package and its
  derivatives
\end{itemize}

\hypertarget{required-computers-and-software}{%
\subsection{Required computers and
software}\label{required-computers-and-software}}

You are required to bring to each class a personal laptop running
Windows, Mac OS X or Linux. You are also required to install the
software \href{http://cran.r-project.org}{R} and the integrated
development environment \href{https://www.rstudio.com}{RStudio}.
Instructions for installing these are available on the Resources page
(see links above).

\hypertarget{outline-of-the-class}{%
\subsection{Outline of the class}\label{outline-of-the-class}}

\begin{longtable}[]{@{}ll@{}}
\toprule
\begin{minipage}[b]{0.17\columnwidth}\raggedright
Date\strut
\end{minipage} & \begin{minipage}[b]{0.77\columnwidth}\raggedright
Topic\strut
\end{minipage}\tabularnewline
\midrule
\endhead
\begin{minipage}[t]{0.17\columnwidth}\raggedright
September 11, 2019\strut
\end{minipage} & \begin{minipage}[t]{0.77\columnwidth}\raggedright
Introduction to R, RStudio and RMarkdown\strut
\end{minipage}\tabularnewline
\begin{minipage}[t]{0.17\columnwidth}\raggedright
September 18, 2019\strut
\end{minipage} & \begin{minipage}[t]{0.77\columnwidth}\raggedright
Data Structures in R (classes 5:30-7, 7-8:30)\strut
\end{minipage}\tabularnewline
\begin{minipage}[t]{0.17\columnwidth}\raggedright
September 25, 2019\strut
\end{minipage} & \begin{minipage}[t]{0.77\columnwidth}\raggedright
R packages, data import/export, munging\strut
\end{minipage}\tabularnewline
\begin{minipage}[t]{0.17\columnwidth}\raggedright
October 02, 2019\strut
\end{minipage} & \begin{minipage}[t]{0.77\columnwidth}\raggedright
Towards analytic data: Data Munging, continued\strut
\end{minipage}\tabularnewline
\begin{minipage}[t]{0.17\columnwidth}\raggedright
October 09, 2019\strut
\end{minipage} & \begin{minipage}[t]{0.77\columnwidth}\raggedright
Data exploration through visualization\strut
\end{minipage}\tabularnewline
\begin{minipage}[t]{0.17\columnwidth}\raggedright
October 16, 2019\strut
\end{minipage} & \begin{minipage}[t]{0.77\columnwidth}\raggedright
More data visualization and RMarkdown\strut
\end{minipage}\tabularnewline
\begin{minipage}[t]{0.17\columnwidth}\raggedright
October 23, 2019\strut
\end{minipage} & \begin{minipage}[t]{0.77\columnwidth}\raggedright
Statistical analyses: Table 1, estimation and confidence intervals, and
more ggplot\strut
\end{minipage}\tabularnewline
\begin{minipage}[t]{0.17\columnwidth}\raggedright
October 30, 2019\strut
\end{minipage} & \begin{minipage}[t]{0.77\columnwidth}\raggedright
Statistical analyses: Classical hypothesis testing and computational
inference\strut
\end{minipage}\tabularnewline
\begin{minipage}[t]{0.17\columnwidth}\raggedright
November 06, 2019\strut
\end{minipage} & \begin{minipage}[t]{0.77\columnwidth}\raggedright
Statistical learning: Regression models\strut
\end{minipage}\tabularnewline
\begin{minipage}[t]{0.17\columnwidth}\raggedright
November 13, 2019\strut
\end{minipage} & \begin{minipage}[t]{0.77\columnwidth}\raggedright
More data munging with \texttt{purrr}: grouping, mapping and functional
programming\strut
\end{minipage}\tabularnewline
\begin{minipage}[t]{0.17\columnwidth}\raggedright
November 20, 2019\strut
\end{minipage} & \begin{minipage}[t]{0.77\columnwidth}\raggedright
Basic bioinformatics: Bioconductor and friends\strut
\end{minipage}\tabularnewline
\begin{minipage}[t]{0.17\columnwidth}\raggedright
November 27, 2019\strut
\end{minipage} & \begin{minipage}[t]{0.77\columnwidth}\raggedright
No class (Thanksgiving)\strut
\end{minipage}\tabularnewline
\begin{minipage}[t]{0.17\columnwidth}\raggedright
December 04, 2019\strut
\end{minipage} & \begin{minipage}[t]{0.77\columnwidth}\raggedright
Statistical learning: Cluster analysis and pattern recognition\strut
\end{minipage}\tabularnewline
\begin{minipage}[t]{0.17\columnwidth}\raggedright
December 11, 2019\strut
\end{minipage} & \begin{minipage}[t]{0.77\columnwidth}\raggedright
Project presentations\strut
\end{minipage}\tabularnewline
\bottomrule
\end{longtable}

\hypertarget{books-and-learning-materials}{%
\subsection{Books and learning
materials}\label{books-and-learning-materials}}

There are no required books for this class. However, we will extensively
refer to a few books available freely online and will serve as reading
material and ongoing reference material for this course.

\begin{enumerate}
\def\labelenumi{\arabic{enumi}.}
\tightlist
\item
  \emph{R for Data Science} {[}R4DS{]} by Hadley Wickham and Garrett
  Grolemund (available \href{https://r4ds.had.co.nz/}{online})
\end{enumerate}

\hypertarget{communication}{%
\subsubsection{Communication}\label{communication}}

This class will communicate primarily via
\href{http://www.slack.com}{Slack}. Please join the BIOF339 Slack
channel using
\href{https://join.slack.com/t/biof339/shared_invite/enQtNzU1Njk5NDc2MDMyLWU5MjkwNzNlMmRlYTAwNDg2MGZjYWQ1YzEzMDUxMzc0MzE1NjZkMDFlY2Y0NTIyMGI2Y2VkODBiYmY0MjBiMzE}{this
link}.

You will see two channels named \emph{wed5-7\_2019} and
\emph{wed7-9\_2019}. Please join the channel corresponding to your
section. I will be using Slack for broadcasting messages, answering
questions and the like. If you have a question, you can directly message
me on Slack if you like. Expect an answer within 24 hours.

\hypertarget{grades}{%
\subsection{Grades}\label{grades}}

Grades will be based on the following requirements:

\begin{enumerate}
\def\labelenumi{\arabic{enumi}.}
\tightlist
\item
  Homeworks, available Friday after class, due by 11:59PM the following
  Tuesday. (50\%)

  \begin{itemize}
  \tightlist
  \item
    No late homeworks, since solutions will be available Wednesday
    mornings
  \item
    I'll score the top 10 homeworks for grade
  \end{itemize}
\item
  Final project: A RMarkdown report/presentation demonstrating an
  end-to-end data analysis in R using your own data, from data ingestion
  to munging to analyses and graphics, with a brief introduction and
  conclusion (20\%)
\item
  Class participation (20\%)
\item
  Completion and submission of class exercises (10\%, marked for
  completion)

  \begin{itemize}
  \tightlist
  \item
    These will need to be in basic RMarkdown, showing the problem and
    the solution. You can add a section for questions here that I can
    address in the following class or online. These will have to be
    submitted before you leave the classroom
  \end{itemize}
\end{enumerate}

\hypertarget{academic-policy-regarding-plagiarism}{%
\subsection{Academic policy regarding
plagiarism}\label{academic-policy-regarding-plagiarism}}

The FAES Graduate School at NIH prides itself on providing quality
educational experiences and upholds the highest level of honesty,
integrity, and mutual respect. It is our policy that cheating,
fabrication or plagiarism by students is\\
not acceptable in any form. If a student is found to be in violation of
any, or all of the below, his/her credits will be forfeited, and he/she
will not be allowed to enroll in future courses or education programs
administered by FAES.

• Cheating is defined as an attempt to give or obtain inappropriate/
unauthorized assistance during any academic exercise, such as during
examination, homework assignment, class presentation.\\
• Fabrication is defined as the falsification of data, information or
citations in any academic materials.\\
• Plagiarism is defined as using the ideas, methods, or written words of
another, without proper acknowledgment and with the intention that they
be taken as the work of the deceiver. These include, but are not limited
to, the use of published articles, paraphrasing, copying someone else's
homework and turning it in as one's own and failing to reference
footnotes. Procuring information from online sources without proper
attribution also constitutes plagiarism.

See
\href{https://faes.org/sites/default/files/Student_Handbook_2016-17.pdfhttps://faes.org/sites/default/files/Student_Handbook_2016-17.pdf}{this
link} for FAES policy on academic plagiarism.

\end{document}
